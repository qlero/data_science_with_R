% Options for packages loaded elsewhere
\PassOptionsToPackage{unicode}{hyperref}
\PassOptionsToPackage{hyphens}{url}
%
\documentclass[
]{article}
\title{Hodgkin-Huxley Piecewise Deterministic Markov Process}
\author{}
\date{\vspace{-2.5em}}

\usepackage{amsmath,amssymb}
\usepackage{lmodern}
\usepackage{iftex}
\ifPDFTeX
  \usepackage[T1]{fontenc}
  \usepackage[utf8]{inputenc}
  \usepackage{textcomp} % provide euro and other symbols
\else % if luatex or xetex
  \usepackage{unicode-math}
  \defaultfontfeatures{Scale=MatchLowercase}
  \defaultfontfeatures[\rmfamily]{Ligatures=TeX,Scale=1}
\fi
% Use upquote if available, for straight quotes in verbatim environments
\IfFileExists{upquote.sty}{\usepackage{upquote}}{}
\IfFileExists{microtype.sty}{% use microtype if available
  \usepackage[]{microtype}
  \UseMicrotypeSet[protrusion]{basicmath} % disable protrusion for tt fonts
}{}
\makeatletter
\@ifundefined{KOMAClassName}{% if non-KOMA class
  \IfFileExists{parskip.sty}{%
    \usepackage{parskip}
  }{% else
    \setlength{\parindent}{0pt}
    \setlength{\parskip}{6pt plus 2pt minus 1pt}}
}{% if KOMA class
  \KOMAoptions{parskip=half}}
\makeatother
\usepackage{xcolor}
\IfFileExists{xurl.sty}{\usepackage{xurl}}{} % add URL line breaks if available
\IfFileExists{bookmark.sty}{\usepackage{bookmark}}{\usepackage{hyperref}}
\hypersetup{
  pdftitle={Hodgkin-Huxley Piecewise Deterministic Markov Process},
  hidelinks,
  pdfcreator={LaTeX via pandoc}}
\urlstyle{same} % disable monospaced font for URLs
\usepackage[margin=1in]{geometry}
\usepackage{color}
\usepackage{fancyvrb}
\newcommand{\VerbBar}{|}
\newcommand{\VERB}{\Verb[commandchars=\\\{\}]}
\DefineVerbatimEnvironment{Highlighting}{Verbatim}{commandchars=\\\{\}}
% Add ',fontsize=\small' for more characters per line
\usepackage{framed}
\definecolor{shadecolor}{RGB}{248,248,248}
\newenvironment{Shaded}{\begin{snugshade}}{\end{snugshade}}
\newcommand{\AlertTok}[1]{\textcolor[rgb]{0.94,0.16,0.16}{#1}}
\newcommand{\AnnotationTok}[1]{\textcolor[rgb]{0.56,0.35,0.01}{\textbf{\textit{#1}}}}
\newcommand{\AttributeTok}[1]{\textcolor[rgb]{0.77,0.63,0.00}{#1}}
\newcommand{\BaseNTok}[1]{\textcolor[rgb]{0.00,0.00,0.81}{#1}}
\newcommand{\BuiltInTok}[1]{#1}
\newcommand{\CharTok}[1]{\textcolor[rgb]{0.31,0.60,0.02}{#1}}
\newcommand{\CommentTok}[1]{\textcolor[rgb]{0.56,0.35,0.01}{\textit{#1}}}
\newcommand{\CommentVarTok}[1]{\textcolor[rgb]{0.56,0.35,0.01}{\textbf{\textit{#1}}}}
\newcommand{\ConstantTok}[1]{\textcolor[rgb]{0.00,0.00,0.00}{#1}}
\newcommand{\ControlFlowTok}[1]{\textcolor[rgb]{0.13,0.29,0.53}{\textbf{#1}}}
\newcommand{\DataTypeTok}[1]{\textcolor[rgb]{0.13,0.29,0.53}{#1}}
\newcommand{\DecValTok}[1]{\textcolor[rgb]{0.00,0.00,0.81}{#1}}
\newcommand{\DocumentationTok}[1]{\textcolor[rgb]{0.56,0.35,0.01}{\textbf{\textit{#1}}}}
\newcommand{\ErrorTok}[1]{\textcolor[rgb]{0.64,0.00,0.00}{\textbf{#1}}}
\newcommand{\ExtensionTok}[1]{#1}
\newcommand{\FloatTok}[1]{\textcolor[rgb]{0.00,0.00,0.81}{#1}}
\newcommand{\FunctionTok}[1]{\textcolor[rgb]{0.00,0.00,0.00}{#1}}
\newcommand{\ImportTok}[1]{#1}
\newcommand{\InformationTok}[1]{\textcolor[rgb]{0.56,0.35,0.01}{\textbf{\textit{#1}}}}
\newcommand{\KeywordTok}[1]{\textcolor[rgb]{0.13,0.29,0.53}{\textbf{#1}}}
\newcommand{\NormalTok}[1]{#1}
\newcommand{\OperatorTok}[1]{\textcolor[rgb]{0.81,0.36,0.00}{\textbf{#1}}}
\newcommand{\OtherTok}[1]{\textcolor[rgb]{0.56,0.35,0.01}{#1}}
\newcommand{\PreprocessorTok}[1]{\textcolor[rgb]{0.56,0.35,0.01}{\textit{#1}}}
\newcommand{\RegionMarkerTok}[1]{#1}
\newcommand{\SpecialCharTok}[1]{\textcolor[rgb]{0.00,0.00,0.00}{#1}}
\newcommand{\SpecialStringTok}[1]{\textcolor[rgb]{0.31,0.60,0.02}{#1}}
\newcommand{\StringTok}[1]{\textcolor[rgb]{0.31,0.60,0.02}{#1}}
\newcommand{\VariableTok}[1]{\textcolor[rgb]{0.00,0.00,0.00}{#1}}
\newcommand{\VerbatimStringTok}[1]{\textcolor[rgb]{0.31,0.60,0.02}{#1}}
\newcommand{\WarningTok}[1]{\textcolor[rgb]{0.56,0.35,0.01}{\textbf{\textit{#1}}}}
\usepackage{longtable,booktabs,array}
\usepackage{calc} % for calculating minipage widths
% Correct order of tables after \paragraph or \subparagraph
\usepackage{etoolbox}
\makeatletter
\patchcmd\longtable{\par}{\if@noskipsec\mbox{}\fi\par}{}{}
\makeatother
% Allow footnotes in longtable head/foot
\IfFileExists{footnotehyper.sty}{\usepackage{footnotehyper}}{\usepackage{footnote}}
\makesavenoteenv{longtable}
\usepackage{graphicx}
\makeatletter
\def\maxwidth{\ifdim\Gin@nat@width>\linewidth\linewidth\else\Gin@nat@width\fi}
\def\maxheight{\ifdim\Gin@nat@height>\textheight\textheight\else\Gin@nat@height\fi}
\makeatother
% Scale images if necessary, so that they will not overflow the page
% margins by default, and it is still possible to overwrite the defaults
% using explicit options in \includegraphics[width, height, ...]{}
\setkeys{Gin}{width=\maxwidth,height=\maxheight,keepaspectratio}
% Set default figure placement to htbp
\makeatletter
\def\fps@figure{htbp}
\makeatother
\setlength{\emergencystretch}{3em} % prevent overfull lines
\providecommand{\tightlist}{%
  \setlength{\itemsep}{0pt}\setlength{\parskip}{0pt}}
\setcounter{secnumdepth}{-\maxdimen} % remove section numbering
\ifLuaTeX
  \usepackage{selnolig}  % disable illegal ligatures
\fi

\begin{document}
\maketitle

\hypertarget{hodgkin-huxley-context}{%
\section{Hodgkin-Huxley Context}\label{hodgkin-huxley-context}}

The Hodgkin-Huxley piecewise deterministic markov process is a model
introduced by Alan Hodgkin and Andrew Huxley in 1952 with the paper
``\href{https://pubmed.ncbi.nlm.nih.gov/12991237/}{A quantitative
description of membrane current and its application to conduction and
excitation in nerve}.'' The Hodgkin-Huxley model targets the initiation
and propagation of an action potential in a neuron (e.g.~current carried
through a neuron with ions penetrating or charging its membrane).

As such, the capacitive current or capacitance flowing through the
neuron's membrane is defined as:

\begin{align}
I &= C_m\frac{dV}{dt} + I_{ion}\\
C_m\frac{dV}{dt} &=- I_{ion}\\
I_{ion} &^= g_{Na}m^3h(V-E_{Na}) + g_Kn^4(V-E_K)+g_L(V-E_L)
\end{align}

Where:

\begin{itemize}
\tightlist
\item
  \(I\) is current per unit area
\item
  \(C_m\) is the capacitance (constant)
\item
  \(V\) is the membrane voltage/potential
\item
  \(I_{ion}\) is the current source caused by the Sodium (Na), Potassium
  (K) and other leaking (L) ions
\item
  \(g_i\) is the electrical conductance of voltage-gated \(i\)-ion
  channels of the gates \(m\), \(n\), and \(h\)
\item
  \(E\) is the equilibrium potential (voltage sources) of the gates
  associated with Sodium (Na), Potassium (K) and other leaking (L) ions
\end{itemize}

The Hodgkin-Huxley equation above is complemented with the following
three differential equations:

\begin{align}
\frac{dm}{dt}&=\alpha_m(V)(1-m)-\beta_m(V)m\\
\frac{dh}{dt}&=\alpha_h(V)(1-h)-\beta_h(V)h\\
\frac{dn}{dt}&=\alpha_n(V)(1-n)-\beta_n(V)n
\end{align}

The rate constants \(\forall i \in \{Na, K, L\},\,\alpha_i\) and
\(\beta_i\) are rate functions estimated by fitting empirical functions
of votage to the experimental data in the Hodgkin-Huxley paper such
that:

In the original model, Hodgkin and Huxley established that there were
three identical activation gates \(m\) and one gate \(h\) to explain the
Sodium (Na) current, and four identical activation gates \(n\) for the
Potassium (K) current.

\hypertarget{simulating-an-example-hodgkin-huxley-pdmp}{%
\subsection{Simulating an example Hodgkin-Huxley
PDMP}\label{simulating-an-example-hodgkin-huxley-pdmp}}

\hypertarget{goal}{%
\subsubsection{GOAL}\label{goal}}

We are interested in implementing the Hodgkin-Huxley model described
above with one hundred gates/doors of each time (\(m\), \(h\), and
\(n\)).

As such, we will first implement the original model shown above, using
\(m\), \(h\), and \(n\) as inputs to modulate the HHPDMP outputs. Then
we will implement the example with 100 doors of each type.

\hypertarget{method}{%
\subsubsection{METHOD}\label{method}}

Setup:

We consider 3 types of doors \(m\), \(h\), and \(n\) such that:

\begin{itemize}
\tightlist
\item
  With \(t\) a given timestep, \(\forall i\in\{m, h, n\},\,N^i\) is the
  number of doors \(X_t^i\) of type \(i\)
\item
  Each door of type \(i\) has two states \(0\) and \(1\) with respective
  transition probabilities:
\end{itemize}

\begin{align}
\forall i\in\{m, h, n\},\,\mathbb{P}_i(0\rightarrow1)&=\alpha_i(V_t)\\
\mathbb{P}_i(1\rightarrow0)&=\beta_i(V_t)\\
\end{align}

We consider \(\hat{m}_t\), \(\hat{h}_t\), and \(\hat{n}_t\) the
respective proportion of open doors of each type such that:

\begin{align}
\hat{m}_t &= \frac{1}{N^m}\underset{i=1}{\overset{N^m}{\sum}}X^{m,i}_t\\
\hat{h}_t &= \frac{1}{N^h}\underset{i=1}{\overset{N^h}{\sum}}X^{h,i}_t\\
\hat{n}_t &= \frac{1}{N^n}\underset{i=1}{\overset{N^n}{\sum}}X^{n,i}_t
\end{align}

The regimes of a Hodgkin-Huxley PDMP corresponds to the \emph{whole}
state \(\big(\frac{k^m}{N^m}, \frac{k^h}{N^h}, \frac{k^n}{N^n}\big)\) or
\(\big(\hat{m}_t, \hat{h}_t, \hat{n}_t\big)\) where
\(\forall i\in\{m, h, n\},\,k^i\) corresponds to the number of open
doors (i.e.~\(X_t^i=1\)) The dynamics of \((\hat{V}_t)\) between the
jumps is:

\begin{align}
C_m\frac{d}{dt}\hat{V}_t &= -g_L(\hat{V}_t - E_L) - \bar{g}_{Na} (\frac{k^m}{N^m})^3 \frac{k^h}{N^h} (\hat{V}_t - E_{Na}) - \bar{g}_K(\frac{k^n}{N^n})^4 (\hat{V}_t - E_K)\\
&= -g_L(\hat{V}_t - E_L) - \bar{g}_{Na} (\hat{m}_t)^3 \hat{h}_t (\hat{V}_t - E_{Na}) - \bar{g}_K(\hat{n}_t)^4 (\hat{V}_t - E_K)
\end{align}

The jumps for each type of door occur at rates:

\begin{align}
\hat{m}^t&\text{ jumps to }
  \begin{cases}
    \hat{m}^t-\frac{1}{N^m} & \text{ at rate }N^m\beta_m(\hat{V}_t)\hat{m}_t\\
    \hat{m}^t+\frac{1}{N^m} & \text{ at rate }N^m\alpha_m(\hat{V}_t)(1-\hat{m}_t)
  \end{cases}\\
\hat{h}^t&\text{ jumps to }
  \begin{cases}
    \hat{h}^t-\frac{1}{N^h} & \text{ at rate }N^h\beta_h(\hat{V}_t)\hat{h}_t\\
    \hat{h}^t+\frac{1}{N^h} & \text{ at rate }N^h\alpha_h(\hat{V}_t)(1-\hat{h}_t)
  \end{cases}\\
\hat{n}^t&\text{ jumps to }
  \begin{cases}
    \hat{n}^t-\frac{1}{N^n} & \text{ at rate }N^n\beta_n(\hat{V}_t)\hat{n}_t\\
    \hat{n}^t+\frac{1}{N^n} & \text{ at rate }N^n\alpha_n(\hat{V}_t)(1-\hat{n}_t)
  \end{cases}\\
\end{align}

With rates \(\alpha_m, \alpha_h, \alpha_n, \beta_m, \beta_h\), and
\(\beta_n\) as stated in the original paper:

(as previously stated)

Simulation method:

We rely on a so-called ``rough algorithm'' such that, given small time
steps \(\delta\), we perform the following update at each time step:

\[\hat{V}_{t+\delta}\approx\hat{V}_t+\delta\frac{d\hat{V}_t}{dt}\]

Where:

\begin{align}
\forall i\in\{m,h,n\},\\
N^i&=100\\
\hat{i}_{t+\delta}&=
  \begin{cases}
    \hat{i}^t & \text{ w/ proba. }1 - \delta.N^i.\big[\beta_i(\hat{V}_t).\hat{i}_t + \alpha_i(\hat{V}_t).(1-\hat{i}_t)\big]\\
    \hat{i}^t-\frac{1}{N^i} & \text{ w/ proba. }\delta.N^i.\beta_i(\hat{V}_t).\hat{i}_t\\
    \hat{i}^t+\frac{1}{N^i} & \text{ w/ proba. }\delta.N^i.\alpha_i(\hat{V}_t).(1-\hat{i}_t)
  \end{cases}\\
\frac{d}{dt}\hat{V}_t &= \frac{-g_L(\hat{V}_t - E_L) - \bar{g}_{Na} (\hat{m}_t)^3 \hat{h}_t (\hat{V}_t - E_{Na}) - \bar{g}_K(\hat{n}_t)^4 (\hat{V}_t - E_K)}{C_m}
\end{align}

And where the \(\alpha\) and \(\beta\) rate functions are taken from the
original paper along with the following parameters:

Function implementations:

\begin{Shaded}
\begin{Highlighting}[]
\CommentTok{\# Declares all intermediary rate functions alpha\_i and beta\_i}

\CommentTok{\# Alpha and Beta rate function from original paper}
\NormalTok{am }\OtherTok{\textless{}{-}} \ControlFlowTok{function}\NormalTok{(V) }\FloatTok{0.1}\SpecialCharTok{*}\NormalTok{(V}\SpecialCharTok{+}\DecValTok{25}\NormalTok{)}\SpecialCharTok{/}\NormalTok{(}\FunctionTok{exp}\NormalTok{((V}\SpecialCharTok{+}\DecValTok{25}\NormalTok{)}\SpecialCharTok{/}\DecValTok{10}\NormalTok{)}\SpecialCharTok{{-}}\DecValTok{1}\NormalTok{)}
\NormalTok{ah }\OtherTok{\textless{}{-}} \ControlFlowTok{function}\NormalTok{(V) }\FloatTok{0.07}\SpecialCharTok{*}\FunctionTok{exp}\NormalTok{(V}\SpecialCharTok{/}\DecValTok{20}\NormalTok{)}
\NormalTok{an }\OtherTok{\textless{}{-}} \ControlFlowTok{function}\NormalTok{(V) }\FloatTok{0.01}\SpecialCharTok{*}\NormalTok{(V}\SpecialCharTok{+}\DecValTok{10}\NormalTok{)}\SpecialCharTok{/}\NormalTok{(}\FunctionTok{exp}\NormalTok{((V}\SpecialCharTok{+}\DecValTok{10}\NormalTok{)}\SpecialCharTok{/}\DecValTok{10}\NormalTok{)}\SpecialCharTok{{-}}\DecValTok{1}\NormalTok{)}
\NormalTok{bm }\OtherTok{\textless{}{-}} \ControlFlowTok{function}\NormalTok{(V) }\DecValTok{4}\SpecialCharTok{*}\FunctionTok{exp}\NormalTok{(V}\SpecialCharTok{/}\DecValTok{18}\NormalTok{)}
\NormalTok{bh }\OtherTok{\textless{}{-}} \ControlFlowTok{function}\NormalTok{(V) }\DecValTok{1}\SpecialCharTok{/}\NormalTok{(}\FunctionTok{exp}\NormalTok{((V}\SpecialCharTok{+}\DecValTok{30}\NormalTok{)}\SpecialCharTok{/}\DecValTok{10}\NormalTok{)}\SpecialCharTok{+}\DecValTok{1}\NormalTok{)}
\NormalTok{bn }\OtherTok{\textless{}{-}} \ControlFlowTok{function}\NormalTok{(V) }\FloatTok{0.125}\SpecialCharTok{*}\FunctionTok{exp}\NormalTok{(V}\SpecialCharTok{/}\DecValTok{80}\NormalTok{)}

\CommentTok{\# Declares the V{-}differential function}
\NormalTok{dvdt }\OtherTok{\textless{}{-}} \ControlFlowTok{function}\NormalTok{(V, Vl, Vna, Vk, gl, gna, gk, m, h, n, C) \{}
  \SpecialCharTok{{-}}\DecValTok{1}\SpecialCharTok{/}\NormalTok{C }\SpecialCharTok{*}\NormalTok{ (gl}\SpecialCharTok{*}\NormalTok{(V}\SpecialCharTok{{-}}\NormalTok{Vl) }\SpecialCharTok{+}\NormalTok{ gna}\SpecialCharTok{*}\NormalTok{m}\SpecialCharTok{\^{}}\DecValTok{3}\SpecialCharTok{*}\NormalTok{h}\SpecialCharTok{*}\NormalTok{(V}\SpecialCharTok{{-}}\NormalTok{Vna) }\SpecialCharTok{+}\NormalTok{ gk}\SpecialCharTok{*}\NormalTok{n}\SpecialCharTok{\^{}}\DecValTok{4}\SpecialCharTok{*}\NormalTok{(V}\SpecialCharTok{{-}}\NormalTok{Vk))}
\NormalTok{\}}

\CommentTok{\# Declares a generic function to update m, h, or n}
\NormalTok{update }\OtherTok{\textless{}{-}} \ControlFlowTok{function}\NormalTok{(i, Ni, V, alpha, beta, delta) \{}
  \CommentTok{\# Draws uniformly a RV, and computes the jump probabilities }
\NormalTok{  proba\_down   }\OtherTok{=}\NormalTok{ Ni}\SpecialCharTok{*}\FunctionTok{beta}\NormalTok{(V)}\SpecialCharTok{*}\NormalTok{i}\SpecialCharTok{*}\NormalTok{delta}
\NormalTok{  proba\_up     }\OtherTok{=}\NormalTok{ Ni}\SpecialCharTok{*}\FunctionTok{alpha}\NormalTok{(V)}\SpecialCharTok{*}\NormalTok{(}\DecValTok{1}\SpecialCharTok{{-}}\NormalTok{i)}\SpecialCharTok{*}\NormalTok{delta}
\NormalTok{  uniform\_draw }\OtherTok{=} \FunctionTok{runif}\NormalTok{(}\DecValTok{1}\NormalTok{, }\DecValTok{0}\NormalTok{, }\DecValTok{1}\NormalTok{)}
  \CommentTok{\# Checks whether a jump occurs or not}
  \ControlFlowTok{if}\NormalTok{ (uniform\_draw }\SpecialCharTok{\textless{}=}\NormalTok{ proba\_down) \{}
    \FunctionTok{return}\NormalTok{(i }\SpecialCharTok{{-}} \DecValTok{1}\SpecialCharTok{/}\NormalTok{Ni)}
\NormalTok{  \} }\ControlFlowTok{else} \ControlFlowTok{if}\NormalTok{ (uniform\_draw }\SpecialCharTok{\textless{}=}\NormalTok{ proba\_down }\SpecialCharTok{+}\NormalTok{ proba\_up) \{}
    \FunctionTok{return}\NormalTok{(i }\SpecialCharTok{+} \DecValTok{1}\SpecialCharTok{/}\NormalTok{Ni)}
\NormalTok{  \} }\ControlFlowTok{else}\NormalTok{ \{}
    \FunctionTok{return}\NormalTok{(i)}
\NormalTok{  \}}
\NormalTok{\}}

\NormalTok{hodgkin\_huxley\_PDMP }\OtherTok{\textless{}{-}} \ControlFlowTok{function}\NormalTok{(}
\NormalTok{  time\_length, timestep,}
\NormalTok{  C, }
\NormalTok{  Vl, Vna, Vk, }
\NormalTok{  m, h, n,}
\NormalTok{  gl, gna, gk,}
\NormalTok{  Nm, Nh, Nn,}
\NormalTok{  am, ah, an, bm, bh, bn,}
\NormalTok{  V, }\AttributeTok{Vr=}\DecValTok{0}
\NormalTok{) \{}
  \DocumentationTok{\#\#\# Hudgkin Huxley PDMP function using a rough algorithm}
  \DocumentationTok{\#\#\# relying on jump probabilities at given small timesteps delta}
  \CommentTok{\# Declares the return space}
\NormalTok{  V }\OtherTok{=} \SpecialCharTok{+}\NormalTok{ V }\SpecialCharTok{{-}}\NormalTok{ Vr }\CommentTok{\# negative depolarization in mV}
\NormalTok{  interval }\OtherTok{=} \FunctionTok{matrix}\NormalTok{(}\FunctionTok{seq}\NormalTok{(}\DecValTok{0}\NormalTok{, time\_length, timestep))}
\NormalTok{  V\_seq    }\OtherTok{=} \FunctionTok{matrix}\NormalTok{(}\FunctionTok{rep}\NormalTok{(V, }\FunctionTok{length}\NormalTok{(interval)))}
\NormalTok{  m\_seq    }\OtherTok{=} \FunctionTok{matrix}\NormalTok{(}\FunctionTok{rep}\NormalTok{(m, }\FunctionTok{length}\NormalTok{(interval)))}
\NormalTok{  h\_seq    }\OtherTok{=} \FunctionTok{matrix}\NormalTok{(}\FunctionTok{rep}\NormalTok{(h, }\FunctionTok{length}\NormalTok{(interval)))}
\NormalTok{  n\_seq    }\OtherTok{=} \FunctionTok{matrix}\NormalTok{(}\FunctionTok{rep}\NormalTok{(n, }\FunctionTok{length}\NormalTok{(interval)))}
  \CommentTok{\# Performs the iterative update per timestep}
  \ControlFlowTok{for}\NormalTok{ (i }\ControlFlowTok{in} \DecValTok{2}\SpecialCharTok{:}\FunctionTok{length}\NormalTok{(interval)) \{}
    \CommentTok{\# Updates m, h, and n}
\NormalTok{    m\_seq[i] }\OtherTok{=} \FunctionTok{update}\NormalTok{(m\_seq[i}\DecValTok{{-}1}\NormalTok{,], Nm, V\_seq[i}\DecValTok{{-}1}\NormalTok{,], am, bm, timestep)}
\NormalTok{    h\_seq[i] }\OtherTok{=} \FunctionTok{update}\NormalTok{(h\_seq[i}\DecValTok{{-}1}\NormalTok{,], Nh, V\_seq[i}\DecValTok{{-}1}\NormalTok{,], ah, bh, timestep)}
\NormalTok{    n\_seq[i] }\OtherTok{=} \FunctionTok{update}\NormalTok{(n\_seq[i}\DecValTok{{-}1}\NormalTok{,], Nn, V\_seq[i}\DecValTok{{-}1}\NormalTok{,], an, bn, timestep)}
    \CommentTok{\# Updates V}
\NormalTok{    V\_seq[i] }\OtherTok{=}\NormalTok{ V\_seq[i}\DecValTok{{-}1}\NormalTok{,] }\SpecialCharTok{+}\NormalTok{ timestep}\SpecialCharTok{*}\FunctionTok{dvdt}\NormalTok{(}
\NormalTok{      V\_seq[i}\DecValTok{{-}1}\NormalTok{,], Vl, Vna, Vk, gl, gna, gk,}
\NormalTok{      m\_seq[i,], h\_seq[i,], n\_seq[i,], C}
\NormalTok{    )}
\NormalTok{  \}}
  \CommentTok{\# Plots the resulting simulation}
  \FunctionTok{par}\NormalTok{(}\AttributeTok{mfrow =} \FunctionTok{c}\NormalTok{(}\DecValTok{2}\NormalTok{, }\DecValTok{2}\NormalTok{), }\AttributeTok{mar=}\FunctionTok{c}\NormalTok{(}\FloatTok{2.8}\NormalTok{,}\FloatTok{2.5}\NormalTok{,}\FloatTok{2.5}\NormalTok{,}\DecValTok{2}\NormalTok{), }\AttributeTok{mgp=}\FunctionTok{c}\NormalTok{(}\FloatTok{1.8}\NormalTok{, }\FloatTok{0.75}\NormalTok{, }\DecValTok{0}\NormalTok{))}
  \CommentTok{\# OF }\AlertTok{NOTE}\CommentTok{: the paper displays the V values as its negative}
  \CommentTok{\# see example graph page 537 here:}
  \CommentTok{\# https://physiology.arizona.edu/sites/default/files/hodgkinhuxley1952\_0.pdf}
  \FunctionTok{plot}\NormalTok{(interval, }\SpecialCharTok{{-}}\NormalTok{V\_seq, }\AttributeTok{type=}\StringTok{"l"}\NormalTok{, }\AttributeTok{col=}\StringTok{\textquotesingle{}red\textquotesingle{}}\NormalTok{, }
       \AttributeTok{xlab=}\StringTok{"mS"}\NormalTok{, }\AttributeTok{ylab=}\StringTok{"{-}V(mV)"}\NormalTok{)}
  \FunctionTok{plot}\NormalTok{(interval, m\_seq, }\AttributeTok{type=}\StringTok{"l"}\NormalTok{, }\AttributeTok{col=}\StringTok{\textquotesingle{}blue\textquotesingle{}}\NormalTok{, }
       \AttributeTok{xlab=}\StringTok{"mS"}\NormalTok{, }\AttributeTok{ylab=}\StringTok{"Proportion open m gates"}\NormalTok{)}
  \FunctionTok{plot}\NormalTok{(interval, h\_seq, }\AttributeTok{type=}\StringTok{"l"}\NormalTok{, }\AttributeTok{col=}\StringTok{\textquotesingle{}green\textquotesingle{}}\NormalTok{, }
       \AttributeTok{xlab=}\StringTok{"mS"}\NormalTok{, }\AttributeTok{ylab=}\StringTok{"Proportion open h gates"}\NormalTok{)}
  \FunctionTok{plot}\NormalTok{(interval, n\_seq, }\AttributeTok{type=}\StringTok{"l"}\NormalTok{, }\AttributeTok{col=}\StringTok{\textquotesingle{}purple\textquotesingle{}}\NormalTok{,}
       \AttributeTok{xlab=}\StringTok{"mS"}\NormalTok{, }\AttributeTok{ylab=}\StringTok{"Proportion open n gates"}\NormalTok{)}
\NormalTok{  title }\OtherTok{=} \FunctionTok{paste}\NormalTok{(}\StringTok{"Simulation results over"}\NormalTok{, time\_length, }\StringTok{"(milliseconds)}\SpecialCharTok{\textbackslash{}n}\StringTok{"}\NormalTok{,}
                \StringTok{"with a timestep of"}\NormalTok{, timestep, }\StringTok{"(milliseconds)"}\NormalTok{)}
  \FunctionTok{mtext}\NormalTok{(title, }\AttributeTok{side =} \DecValTok{1}\NormalTok{, }\AttributeTok{line =} \SpecialCharTok{{-}}\DecValTok{29}\NormalTok{, }\AttributeTok{outer=}\ConstantTok{TRUE}\NormalTok{)}
  \CommentTok{\# Returns}
  \FunctionTok{return}\NormalTok{(}\FunctionTok{list}\NormalTok{(}
    \StringTok{"timesteps"}\OtherTok{=}\NormalTok{interval, }
    \StringTok{"V"}\OtherTok{=}\NormalTok{V\_seq, }\StringTok{"{-}V"}\OtherTok{=}\SpecialCharTok{{-}}\NormalTok{V\_seq,}
    \StringTok{"m"}\OtherTok{=}\NormalTok{m\_seq, }\StringTok{"h"}\OtherTok{=}\NormalTok{h\_seq, }\StringTok{"n"}\OtherTok{=}\NormalTok{n\_seq}
\NormalTok{    )}
\NormalTok{  )}
\NormalTok{\}}
\end{Highlighting}
\end{Shaded}

Of note, up until now, we have shown the modeling of \(V\) (in microVolt
\(mV\)). In the original paper, the authors then switch to data
visualizing the modeling of \(V\) via its inverse \(-V\) such as in page
537
\href{https://physiology.arizona.edu/sites/default/files/hodgkinhuxley1952_0.pdf}{here}
and reproduced below:

As such, the function declared above returns both \(V\) and \(-V\) but
will only display graphically the latter to keep with the original
paper's format.

\hypertarget{results---simulation-with-the-rough-algorithm}{%
\subsubsection{RESULTS - Simulation with the Rough
algorithm}\label{results---simulation-with-the-rough-algorithm}}

Declaring the general parameter space:

\begin{longtable}[]{@{}lll@{}}
\toprule
Parameters & Value & Note \\
\midrule
\endhead
length & 100 & milliseconds \\
timestep \(\delta\) & 0.01 & milliseconds \\
\(C_m\) & 1 & \emph{taken from original paper} \\
\(V_l\) & -10.615 & \emph{taken from original paper} \\
\(V_{Na}\) & -115 & \emph{taken from original paper} \\
\(V_k\) & 12 & \emph{taken from original paper} \\
\(m\) & 0.7 & arbitrary choice \\
\(h\) & 0.3 & arbitrary choice \\
\(n\) & 0.1 & arbitrary choice \\
\(g_l\) & 0.3 & \emph{taken from original paper} \\
\(g_{Na}\) & 120 & \emph{taken from original paper} \\
\(g_k\) & 36 & \emph{taken from original paper} \\
\(Nm\) & 100 & arbitrary choice \\
\(Nh\) & 100 & arbitrary choice \\
\(Nn\) & 100 & arbitrary choice \\
\(V\) & 30 & \emph{taken from original paper} \\
\(V_r\) & 0 & \emph{taken from original paper} \\
\bottomrule
\end{longtable}

\begin{Shaded}
\begin{Highlighting}[]
\CommentTok{\# Declares parameters}
\NormalTok{time\_length }\OtherTok{=} \DecValTok{100}     \CommentTok{\# arbitrary choice}
\NormalTok{timestep    }\OtherTok{=} \FloatTok{0.01}    \CommentTok{\# arbitrary choice}
\NormalTok{C           }\OtherTok{=} \DecValTok{1}       \CommentTok{\# paper OG value (resting potential 0)}
\CommentTok{\# Conductance levels}
\NormalTok{Vl          }\OtherTok{=} \SpecialCharTok{{-}}\FloatTok{10.613} \CommentTok{\# paper OG value (resting potential 0)}
\NormalTok{Vna         }\OtherTok{=} \SpecialCharTok{{-}}\DecValTok{115}    \CommentTok{\# paper OG value (resting potential 0)}
\NormalTok{Vk          }\OtherTok{=} \DecValTok{12}      \CommentTok{\# paper OG value (resting potential 0)}
\CommentTok{\# Initial open proportions}
\NormalTok{m           }\OtherTok{=} \FloatTok{0.7}     \CommentTok{\# arbitrary choice}
\NormalTok{h           }\OtherTok{=} \FloatTok{0.3}     \CommentTok{\# arbitrary choice}
\NormalTok{n           }\OtherTok{=} \FloatTok{0.1}     \CommentTok{\# arbitrary choice}
\CommentTok{\# Conductance values}
\NormalTok{gl          }\OtherTok{=} \FloatTok{0.3}     \CommentTok{\# paper OG value (resting potential 0) }
\NormalTok{gna         }\OtherTok{=} \DecValTok{120}     \CommentTok{\# paper OG value (resting potential 0)}
\NormalTok{gk          }\OtherTok{=} \DecValTok{36}      \CommentTok{\# paper OG value (resting potential 0)}
\CommentTok{\# number of doors}
\NormalTok{Nm          }\OtherTok{=} \DecValTok{100}     \CommentTok{\# arbitrary choice}
\NormalTok{Nh          }\OtherTok{=} \DecValTok{100}     \CommentTok{\# arbitrary choice}
\NormalTok{Nn          }\OtherTok{=} \DecValTok{100}     \CommentTok{\# arbitrary choice}
\CommentTok{\# Membrane displacement potential (negative depolarization)}
\NormalTok{V           }\OtherTok{=} \DecValTok{30}      \CommentTok{\# arbitrary choice (original paper has V starting around 30)}
\CommentTok{\# For V, see page 536 here:}
\CommentTok{\# https://physiology.arizona.edu/sites/default/files/hodgkinhuxley1952\_0.pdf}
\NormalTok{Vr          }\OtherTok{=} \DecValTok{0}       \CommentTok{\# paper OG value (resting potential 0)}
\end{Highlighting}
\end{Shaded}

Varying the timelength with a given timestep of 0.01:

\begin{Shaded}
\begin{Highlighting}[]
\NormalTok{simulation }\OtherTok{=} \FunctionTok{hodgkin\_huxley\_PDMP}\NormalTok{(}
\NormalTok{  time\_length, timestep,}
\NormalTok{  C, }
\NormalTok{  Vl, Vna, Vk, }
\NormalTok{  m, h, n,}
\NormalTok{  gl, gna, gk,}
\NormalTok{  Nm, Nh, Nn,}
\NormalTok{  am, ah, an, bm, bh, bn,}
\NormalTok{  V, Vr}
\NormalTok{)}
\end{Highlighting}
\end{Shaded}

\includegraphics[width=1\linewidth]{hodgkin_huxley_files/figure-latex/results-1}

\begin{Shaded}
\begin{Highlighting}[]
\CommentTok{\# Declares parameters}
\NormalTok{time\_length }\OtherTok{=} \DecValTok{1000} \CommentTok{\# arbitrary choice}

\NormalTok{simulation }\OtherTok{=} \FunctionTok{hodgkin\_huxley\_PDMP}\NormalTok{(}
\NormalTok{  time\_length, timestep,}
\NormalTok{  C, }
\NormalTok{  Vl, Vna, Vk, }
\NormalTok{  m, h, n,}
\NormalTok{  gl, gna, gk,}
\NormalTok{  Nm, Nh, Nn,}
\NormalTok{  am, ah, an, bm, bh, bn,}
\NormalTok{  V, Vr}
\NormalTok{)}
\end{Highlighting}
\end{Shaded}

\includegraphics[width=1\linewidth]{hodgkin_huxley_files/figure-latex/results_2-1}

\begin{Shaded}
\begin{Highlighting}[]
\CommentTok{\# Declares parameters}
\NormalTok{time\_length }\OtherTok{=} \DecValTok{10} \CommentTok{\# arbitrary choice}

\NormalTok{simulation }\OtherTok{=} \FunctionTok{hodgkin\_huxley\_PDMP}\NormalTok{(}
\NormalTok{  time\_length, timestep,}
\NormalTok{  C, }
\NormalTok{  Vl, Vna, Vk, }
\NormalTok{  m, h, n,}
\NormalTok{  gl, gna, gk,}
\NormalTok{  Nm, Nh, Nn,}
\NormalTok{  am, ah, an, bm, bh, bn,}
\NormalTok{  V, Vr}
\NormalTok{)}
\end{Highlighting}
\end{Shaded}

\includegraphics[width=1\linewidth]{hodgkin_huxley_files/figure-latex/results_3-1}

Varying the timelength with a given timestep of 0.001:

\begin{Shaded}
\begin{Highlighting}[]
\CommentTok{\# Declares parameters}
\NormalTok{time\_length }\OtherTok{=} \DecValTok{100}   \CommentTok{\# arbitrary choice}
\NormalTok{timestep    }\OtherTok{=} \FloatTok{0.001} \CommentTok{\# arbitrary choice}

\NormalTok{simulation }\OtherTok{=} \FunctionTok{hodgkin\_huxley\_PDMP}\NormalTok{(}
\NormalTok{  time\_length, timestep,}
\NormalTok{  C, }
\NormalTok{  Vl, Vna, Vk, }
\NormalTok{  m, h, n,}
\NormalTok{  gl, gna, gk,}
\NormalTok{  Nm, Nh, Nn,}
\NormalTok{  am, ah, an, bm, bh, bn,}
\NormalTok{  V, Vr}
\NormalTok{)}
\end{Highlighting}
\end{Shaded}

\includegraphics[width=1\linewidth]{hodgkin_huxley_files/figure-latex/results_4-1}

\begin{Shaded}
\begin{Highlighting}[]
\CommentTok{\# Declares parameters}
\NormalTok{time\_length }\OtherTok{=} \DecValTok{1000}  \CommentTok{\# arbitrary choice}

\NormalTok{simulation }\OtherTok{=} \FunctionTok{hodgkin\_huxley\_PDMP}\NormalTok{(}
\NormalTok{  time\_length, timestep,}
\NormalTok{  C, }
\NormalTok{  Vl, Vna, Vk, }
\NormalTok{  m, h, n,}
\NormalTok{  gl, gna, gk,}
\NormalTok{  Nm, Nh, Nn,}
\NormalTok{  am, ah, an, bm, bh, bn,}
\NormalTok{  V, Vr}
\NormalTok{)}
\end{Highlighting}
\end{Shaded}

\includegraphics[width=1\linewidth]{hodgkin_huxley_files/figure-latex/results_5-1}

\begin{Shaded}
\begin{Highlighting}[]
\CommentTok{\# Declares parameters}
\NormalTok{time\_length }\OtherTok{=} \DecValTok{10}    \CommentTok{\# arbitrary choice}

\NormalTok{simulation }\OtherTok{=} \FunctionTok{hodgkin\_huxley\_PDMP}\NormalTok{(}
\NormalTok{  time\_length, timestep,}
\NormalTok{  C, }
\NormalTok{  Vl, Vna, Vk, }
\NormalTok{  m, h, n,}
\NormalTok{  gl, gna, gk,}
\NormalTok{  Nm, Nh, Nn,}
\NormalTok{  am, ah, an, bm, bh, bn,}
\NormalTok{  V, Vr}
\NormalTok{)}
\end{Highlighting}
\end{Shaded}

\includegraphics[width=1\linewidth]{hodgkin_huxley_files/figure-latex/results_6-1}

Varying the timelength with a given timestep of 0.1:

\begin{Shaded}
\begin{Highlighting}[]
\CommentTok{\# Declares parameters}
\NormalTok{time\_length }\OtherTok{=} \DecValTok{100}   \CommentTok{\# arbitrary choice}
\NormalTok{timestep    }\OtherTok{=} \FloatTok{0.1} \CommentTok{\# arbitrary choice}

\NormalTok{simulation }\OtherTok{=} \FunctionTok{hodgkin\_huxley\_PDMP}\NormalTok{(}
\NormalTok{  time\_length, timestep,}
\NormalTok{  C, }
\NormalTok{  Vl, Vna, Vk, }
\NormalTok{  m, h, n,}
\NormalTok{  gl, gna, gk,}
\NormalTok{  Nm, Nh, Nn,}
\NormalTok{  am, ah, an, bm, bh, bn,}
\NormalTok{  V, Vr}
\NormalTok{)}
\end{Highlighting}
\end{Shaded}

\includegraphics[width=1\linewidth]{hodgkin_huxley_files/figure-latex/results_7-1}

\begin{Shaded}
\begin{Highlighting}[]
\CommentTok{\# Declares parameters}
\NormalTok{time\_length }\OtherTok{=} \DecValTok{1000}   \CommentTok{\# arbitrary choice}

\NormalTok{simulation }\OtherTok{=} \FunctionTok{hodgkin\_huxley\_PDMP}\NormalTok{(}
\NormalTok{  time\_length, timestep,}
\NormalTok{  C, }
\NormalTok{  Vl, Vna, Vk, }
\NormalTok{  m, h, n,}
\NormalTok{  gl, gna, gk,}
\NormalTok{  Nm, Nh, Nn,}
\NormalTok{  am, ah, an, bm, bh, bn,}
\NormalTok{  V, Vr}
\NormalTok{)}
\end{Highlighting}
\end{Shaded}

\includegraphics[width=1\linewidth]{hodgkin_huxley_files/figure-latex/results_8-1}

\begin{Shaded}
\begin{Highlighting}[]
\CommentTok{\# Declares parameters}
\NormalTok{time\_length }\OtherTok{=} \DecValTok{10}   \CommentTok{\# arbitrary choice}

\NormalTok{simulation }\OtherTok{=} \FunctionTok{hodgkin\_huxley\_PDMP}\NormalTok{(}
\NormalTok{  time\_length, timestep,}
\NormalTok{  C, }
\NormalTok{  Vl, Vna, Vk, }
\NormalTok{  m, h, n,}
\NormalTok{  gl, gna, gk,}
\NormalTok{  Nm, Nh, Nn,}
\NormalTok{  am, ah, an, bm, bh, bn,}
\NormalTok{  V, Vr}
\NormalTok{)}
\end{Highlighting}
\end{Shaded}

\includegraphics[width=1\linewidth]{hodgkin_huxley_files/figure-latex/results_9-1}

Varying gate conductance parameters:

\begin{Shaded}
\begin{Highlighting}[]
\CommentTok{\# Declares parameters}
\NormalTok{time\_length }\OtherTok{=} \DecValTok{100}     \CommentTok{\# arbitrary choice}
\NormalTok{timestep    }\OtherTok{=} \FloatTok{0.01}    \CommentTok{\# arbitrary choice}
\CommentTok{\# Initial open proportions}
\NormalTok{m           }\OtherTok{=} \DecValTok{0}    \CommentTok{\# arbitrary choice}
\NormalTok{h           }\OtherTok{=} \DecValTok{0}    \CommentTok{\# arbitrary choice}
\NormalTok{n           }\OtherTok{=} \DecValTok{0}    \CommentTok{\# arbitrary choice}

\NormalTok{simulation }\OtherTok{=} \FunctionTok{hodgkin\_huxley\_PDMP}\NormalTok{(}
\NormalTok{  time\_length, timestep,}
\NormalTok{  C, }
\NormalTok{  Vl, Vna, Vk, }
\NormalTok{  m, h, n,}
\NormalTok{  gl, gna, gk,}
\NormalTok{  Nm, Nh, Nn,}
\NormalTok{  am, ah, an, bm, bh, bn,}
\NormalTok{  V, Vr}
\NormalTok{)}
\end{Highlighting}
\end{Shaded}

\includegraphics[width=1\linewidth]{hodgkin_huxley_files/figure-latex/results_gates_1-1}

\begin{Shaded}
\begin{Highlighting}[]
\CommentTok{\# Declares parameters}
\NormalTok{time\_length }\OtherTok{=} \DecValTok{100}     \CommentTok{\# arbitrary choice}
\NormalTok{timestep    }\OtherTok{=} \FloatTok{0.01}    \CommentTok{\# arbitrary choice}
\CommentTok{\# Initial open proportions}
\NormalTok{m           }\OtherTok{=} \DecValTok{1}    \CommentTok{\# arbitrary choice}
\NormalTok{h           }\OtherTok{=} \DecValTok{1}    \CommentTok{\# arbitrary choice}
\NormalTok{n           }\OtherTok{=} \DecValTok{1}    \CommentTok{\# arbitrary choice}

\NormalTok{simulation }\OtherTok{=} \FunctionTok{hodgkin\_huxley\_PDMP}\NormalTok{(}
\NormalTok{  time\_length, timestep,}
\NormalTok{  C, }
\NormalTok{  Vl, Vna, Vk, }
\NormalTok{  m, h, n,}
\NormalTok{  gl, gna, gk,}
\NormalTok{  Nm, Nh, Nn,}
\NormalTok{  am, ah, an, bm, bh, bn,}
\NormalTok{  V, Vr}
\NormalTok{)}
\end{Highlighting}
\end{Shaded}

\includegraphics[width=1\linewidth]{hodgkin_huxley_files/figure-latex/results_gates_12-1}

\begin{Shaded}
\begin{Highlighting}[]
\CommentTok{\# Declares parameters}
\NormalTok{time\_length }\OtherTok{=} \DecValTok{100}     \CommentTok{\# arbitrary choice}
\NormalTok{timestep    }\OtherTok{=} \FloatTok{0.01}    \CommentTok{\# arbitrary choice}
\CommentTok{\# Initial open proportions}
\NormalTok{m           }\OtherTok{=} \FloatTok{0.7}     \CommentTok{\# arbitrary choice}
\NormalTok{h           }\OtherTok{=} \FloatTok{0.3}     \CommentTok{\# arbitrary choice}
\NormalTok{n           }\OtherTok{=} \FloatTok{0.1}     \CommentTok{\# arbitrary choice}
\CommentTok{\# Conductance levels}
\NormalTok{Vl          }\OtherTok{=} \SpecialCharTok{{-}}\DecValTok{100}    
\NormalTok{Vna         }\OtherTok{=} \SpecialCharTok{{-}}\DecValTok{100}    
\NormalTok{Vk          }\OtherTok{=} \DecValTok{100}     

\NormalTok{simulation }\OtherTok{=} \FunctionTok{hodgkin\_huxley\_PDMP}\NormalTok{(}
\NormalTok{  time\_length, timestep,}
\NormalTok{  C, }
\NormalTok{  Vl, Vna, Vk, }
\NormalTok{  m, h, n,}
\NormalTok{  gl, gna, gk,}
\NormalTok{  Nm, Nh, Nn,}
\NormalTok{  am, ah, an, bm, bh, bn,}
\NormalTok{  V, Vr}
\NormalTok{)}
\end{Highlighting}
\end{Shaded}

\includegraphics[width=1\linewidth]{hodgkin_huxley_files/figure-latex/results_gates_2-1}

\begin{Shaded}
\begin{Highlighting}[]
\CommentTok{\# Declares parameters}
\NormalTok{time\_length }\OtherTok{=} \DecValTok{100}     \CommentTok{\# arbitrary choice}
\NormalTok{timestep    }\OtherTok{=} \FloatTok{0.01}    \CommentTok{\# arbitrary choice}
\CommentTok{\# Conductance levels}
\NormalTok{Vl          }\OtherTok{=} \DecValTok{0}       \CommentTok{\# arbitrary choice}
\NormalTok{Vna         }\OtherTok{=} \SpecialCharTok{{-}}\DecValTok{10}     \CommentTok{\# arbitrary choice}
\NormalTok{Vk          }\OtherTok{=} \DecValTok{10}      \CommentTok{\# arbitrary choice}

\NormalTok{simulation }\OtherTok{=} \FunctionTok{hodgkin\_huxley\_PDMP}\NormalTok{(}
\NormalTok{  time\_length, timestep,}
\NormalTok{  C, }
\NormalTok{  Vl, Vna, Vk, }
\NormalTok{  m, h, n,}
\NormalTok{  gl, gna, gk,}
\NormalTok{  Nm, Nh, Nn,}
\NormalTok{  am, ah, an, bm, bh, bn,}
\NormalTok{  V, Vr}
\NormalTok{)}
\end{Highlighting}
\end{Shaded}

\includegraphics[width=1\linewidth]{hodgkin_huxley_files/figure-latex/results_gates_3-1}

\begin{Shaded}
\begin{Highlighting}[]
\CommentTok{\# Declares parameters}
\NormalTok{time\_length }\OtherTok{=} \DecValTok{100}     \CommentTok{\# arbitrary choice}
\NormalTok{timestep    }\OtherTok{=} \FloatTok{0.01}    \CommentTok{\# arbitrary choice}
\CommentTok{\# Conductance levels}
\NormalTok{Vl          }\OtherTok{=} \FloatTok{10.613}  \CommentTok{\# arbitrary choice}
\NormalTok{Vna         }\OtherTok{=} \DecValTok{115}     \CommentTok{\# arbitrary choice}
\NormalTok{Vk          }\OtherTok{=} \SpecialCharTok{{-}}\DecValTok{12}     \CommentTok{\# arbitrary choice}

\NormalTok{simulation }\OtherTok{=} \FunctionTok{hodgkin\_huxley\_PDMP}\NormalTok{(}
\NormalTok{  time\_length, timestep,}
\NormalTok{  C, }
\NormalTok{  Vl, Vna, Vk, }
\NormalTok{  m, h, n,}
\NormalTok{  gl, gna, gk,}
\NormalTok{  Nm, Nh, Nn,}
\NormalTok{  am, ah, an, bm, bh, bn,}
\NormalTok{  V, Vr}
\NormalTok{)}
\end{Highlighting}
\end{Shaded}

\includegraphics[width=1\linewidth]{hodgkin_huxley_files/figure-latex/results_gates_4-1}

\hypertarget{comments}{%
\subsubsection{COMMENTS}\label{comments}}

Starting with the following baseline set of parameters:

\begin{longtable}[]{@{}lll@{}}
\toprule
Parameters & Value & Note \\
\midrule
\endhead
length & 100 & milliseconds \\
timestep \(\delta\) & 0.01 & milliseconds \\
\(C_m\) & 1 & \emph{taken from original paper} \\
\(V_l\) & -10.615 & \emph{taken from original paper} \\
\(V_{Na}\) & -115 & \emph{taken from original paper} \\
\(V_k\) & 12 & \emph{taken from original paper} \\
\(m\) & 0.7 & arbitrary choice \\
\(h\) & 0.3 & arbitrary choice \\
\(n\) & 0.1 & arbitrary choice \\
\(g_l\) & 0.3 & \emph{taken from original paper} \\
\(g_{Na}\) & 120 & \emph{taken from original paper} \\
\(g_k\) & 36 & \emph{taken from original paper} \\
\(Nm\) & 100 & arbitrary choice \\
\(Nh\) & 100 & arbitrary choice \\
\(Nn\) & 100 & arbitrary choice \\
\(V\) & 30 & \emph{taken from original paper} \\
\(V_r\) & 0 & \emph{taken from original paper} \\
\bottomrule
\end{longtable}

We can observe the following:

\begin{itemize}
\tightlist
\item
  Given a simulation timestep \(\delta\) of 0.01 or 0.001 milliseconds,
  we find that we have, in general, a simulated spike every 20 to 30
  milliseconds.
\item
  Increasing the timestep \(\delta\) to 0.1 milliseconds alters the
  behavior of the rough simulation -- indicating that the function might
  not work at high timestep resolutions:

  \begin{itemize}
  \tightlist
  \item
    Only one spike is generated
  \item
    The spike's shape differs from other setups with a smaller/finer
    \(\delta\)
  \end{itemize}
\item
  Given a simulation timestep \(\delta\) of 0.01 and a simulated length
  of \(100\) milliseconds, we find:

  \begin{itemize}
  \tightlist
  \item
    Setting the initial proportions \(m\), \(h\), and \(n\) to either 0
    or 1 does not seem to affect the general behavior of the simulation
    after a few steps as the proportions' behavior seem to converge back
    to what was observed in the very first simulation
  \item
    Modifying the ion capacitance \(V_l\), \(V_{Na}\), and \(V_k\):
    strongly affects the simulation's behavior

    \begin{itemize}
    \tightlist
    \item
      Increasing each value \(V_l\), \(V_k\), \(V_{Na}\) by an order of
      magnitude seems to double or triple the number of generated spikes
    \item
      Reducing these values or nullifying them leads to an absence of
      spikes and a capacitance \(V\) hovering around 0
    \item
      Inverting the values \(V_l\), \(V_k\), \(V_{Na}\) as stated in the
      original paper also leads to an absence of spikes and a
      capacitance \(V\) converging towards 0
    \end{itemize}
  \end{itemize}
\end{itemize}

\hypertarget{implementation-of-the-differential-model-with-the-simecol-library}{%
\subsubsection{IMPLEMENTATION OF THE DIFFERENTIAL MODEL WITH THE SIMECOL
LIBRARY}\label{implementation-of-the-differential-model-with-the-simecol-library}}

Exported from the
\href{https://www.r-bloggers.com/2012/06/hodgkin-huxley-model-in-r/}{R-bloggers}
website, it is possible to perform a Hodgkin-Huxley dynamic simulation
(here with Ordinary Differential Equations (ODE) from the
\href{http://simecol.r-forge.r-project.org/}{simecol library}).

We note that the provided rate functions \(\alpha\) and \(\beta\)
differs from the original paper.

\begin{Shaded}
\begin{Highlighting}[]
\CommentTok{\# install.packages("simecol")}
\FunctionTok{library}\NormalTok{(simecol)}
\end{Highlighting}
\end{Shaded}

\begin{Shaded}
\begin{Highlighting}[]
\DocumentationTok{\#\# Hodkin{-}Huxley model}
\NormalTok{HH }\OtherTok{\textless{}{-}} \FunctionTok{odeModel}\NormalTok{(}
\AttributeTok{main =} \ControlFlowTok{function}\NormalTok{(time, init, parms) \{}
  \FunctionTok{with}\NormalTok{(}\FunctionTok{as.list}\NormalTok{(}\FunctionTok{c}\NormalTok{(init, parms)),\{}

\NormalTok{    am }\OtherTok{\textless{}{-}} \ControlFlowTok{function}\NormalTok{(v) }\FloatTok{0.1}\SpecialCharTok{*}\NormalTok{(v}\SpecialCharTok{+}\DecValTok{40}\NormalTok{)}\SpecialCharTok{/}\NormalTok{(}\DecValTok{1}\SpecialCharTok{{-}}\FunctionTok{exp}\NormalTok{(}\SpecialCharTok{{-}}\NormalTok{(v}\SpecialCharTok{+}\DecValTok{40}\NormalTok{)}\SpecialCharTok{/}\DecValTok{10}\NormalTok{))}
\NormalTok{    bm }\OtherTok{\textless{}{-}} \ControlFlowTok{function}\NormalTok{(v) }\DecValTok{4}\SpecialCharTok{*}\FunctionTok{exp}\NormalTok{(}\SpecialCharTok{{-}}\NormalTok{(v}\SpecialCharTok{+}\DecValTok{65}\NormalTok{)}\SpecialCharTok{/}\DecValTok{18}\NormalTok{)}
\NormalTok{    ah }\OtherTok{\textless{}{-}} \ControlFlowTok{function}\NormalTok{(v) }\FloatTok{0.07}\SpecialCharTok{*}\FunctionTok{exp}\NormalTok{(}\SpecialCharTok{{-}}\NormalTok{(v}\SpecialCharTok{+}\DecValTok{65}\NormalTok{)}\SpecialCharTok{/}\DecValTok{20}\NormalTok{)}
\NormalTok{    bh }\OtherTok{\textless{}{-}} \ControlFlowTok{function}\NormalTok{(v) }\DecValTok{1}\SpecialCharTok{/}\NormalTok{(}\DecValTok{1}\SpecialCharTok{+}\FunctionTok{exp}\NormalTok{(}\SpecialCharTok{{-}}\NormalTok{(v}\SpecialCharTok{+}\DecValTok{35}\NormalTok{)}\SpecialCharTok{/}\DecValTok{10}\NormalTok{))}
\NormalTok{    an }\OtherTok{\textless{}{-}} \ControlFlowTok{function}\NormalTok{(v) }\FloatTok{0.01}\SpecialCharTok{*}\NormalTok{(v}\SpecialCharTok{+}\DecValTok{55}\NormalTok{)}\SpecialCharTok{/}\NormalTok{(}\DecValTok{1}\SpecialCharTok{{-}}\FunctionTok{exp}\NormalTok{(}\SpecialCharTok{{-}}\NormalTok{(v}\SpecialCharTok{+}\DecValTok{55}\NormalTok{)}\SpecialCharTok{/}\DecValTok{10}\NormalTok{))}
\NormalTok{    bn }\OtherTok{\textless{}{-}} \ControlFlowTok{function}\NormalTok{(v) }\FloatTok{0.125}\SpecialCharTok{*}\FunctionTok{exp}\NormalTok{(}\SpecialCharTok{{-}}\NormalTok{(v}\SpecialCharTok{+}\DecValTok{65}\NormalTok{)}\SpecialCharTok{/}\DecValTok{80}\NormalTok{)}
    
\NormalTok{    dv }\OtherTok{\textless{}{-}}\NormalTok{ (I }\SpecialCharTok{{-}}\NormalTok{ gna}\SpecialCharTok{*}\NormalTok{h}\SpecialCharTok{*}\NormalTok{(v}\SpecialCharTok{{-}}\NormalTok{Ena)}\SpecialCharTok{*}\NormalTok{m}\SpecialCharTok{\^{}}\DecValTok{3}\SpecialCharTok{{-}}\NormalTok{gk}\SpecialCharTok{*}\NormalTok{(v}\SpecialCharTok{{-}}\NormalTok{Ek)}\SpecialCharTok{*}\NormalTok{n}\SpecialCharTok{\^{}}\DecValTok{4}\SpecialCharTok{{-}}\NormalTok{gl}\SpecialCharTok{*}\NormalTok{(v}\SpecialCharTok{{-}}\NormalTok{El))}\SpecialCharTok{/}\NormalTok{C}
\NormalTok{    dm }\OtherTok{\textless{}{-}} \FunctionTok{am}\NormalTok{(v)}\SpecialCharTok{*}\NormalTok{(}\DecValTok{1}\SpecialCharTok{{-}}\NormalTok{m)}\SpecialCharTok{{-}}\FunctionTok{bm}\NormalTok{(v)}\SpecialCharTok{*}\NormalTok{m}
\NormalTok{    dh }\OtherTok{\textless{}{-}} \FunctionTok{ah}\NormalTok{(v)}\SpecialCharTok{*}\NormalTok{(}\DecValTok{1}\SpecialCharTok{{-}}\NormalTok{h)}\SpecialCharTok{{-}}\FunctionTok{bh}\NormalTok{(v)}\SpecialCharTok{*}\NormalTok{h}
\NormalTok{    dn }\OtherTok{\textless{}{-}} \FunctionTok{an}\NormalTok{(v)}\SpecialCharTok{*}\NormalTok{(}\DecValTok{1}\SpecialCharTok{{-}}\NormalTok{n)}\SpecialCharTok{{-}}\FunctionTok{bn}\NormalTok{(v)}\SpecialCharTok{*}\NormalTok{n}
    
    \FunctionTok{return}\NormalTok{(}\FunctionTok{list}\NormalTok{(}\FunctionTok{c}\NormalTok{(dv, dm, dh, dn)))}
\NormalTok{  \})}
\NormalTok{  \},}
  \DocumentationTok{\#\# Set parameters}
  \AttributeTok{parms =} \FunctionTok{c}\NormalTok{(}\AttributeTok{Ena=}\DecValTok{50}\NormalTok{, }\AttributeTok{Ek=}\SpecialCharTok{{-}}\DecValTok{77}\NormalTok{, }\AttributeTok{El=}\SpecialCharTok{{-}}\FloatTok{54.4}\NormalTok{, }\AttributeTok{gna=}\DecValTok{120}\NormalTok{, }\AttributeTok{gk=}\DecValTok{36}\NormalTok{, }\AttributeTok{gl=}\FloatTok{0.3}\NormalTok{, }\AttributeTok{C=}\DecValTok{1}\NormalTok{, }\AttributeTok{I=}\DecValTok{0}\NormalTok{),}
  \DocumentationTok{\#\# Set integrations times}
  \AttributeTok{times =} \FunctionTok{c}\NormalTok{(}\AttributeTok{from=}\DecValTok{0}\NormalTok{, }\AttributeTok{to=}\DecValTok{40}\NormalTok{, }\AttributeTok{by =} \FloatTok{0.25}\NormalTok{),}
  \DocumentationTok{\#\# Set initial state}
  \AttributeTok{init =} \FunctionTok{c}\NormalTok{(}\AttributeTok{v=}\SpecialCharTok{{-}}\DecValTok{65}\NormalTok{, }\AttributeTok{m=}\FloatTok{0.052}\NormalTok{, }\AttributeTok{h=}\FloatTok{0.596}\NormalTok{, }\AttributeTok{n=}\FloatTok{0.317}\NormalTok{),}
  \AttributeTok{solver =} \StringTok{"lsoda"}
\NormalTok{)}
\end{Highlighting}
\end{Shaded}

\begin{Shaded}
\begin{Highlighting}[]
\NormalTok{HH }\OtherTok{\textless{}{-}} \FunctionTok{sim}\NormalTok{(HH)}
\FunctionTok{plot}\NormalTok{(HH)}
\end{Highlighting}
\end{Shaded}

\includegraphics{hodgkin_huxley_files/figure-latex/run_model-1.pdf}

\end{document}
