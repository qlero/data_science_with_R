%Para slide em "Wide-Screen" usar:
\documentclass[aspectratio=43,10pt]{beamer} 

%Para slide "quadrado" usar:
%\documentclass{beamer} 

\usepackage{tikz}

%Elaborado por Mateus Moro Lumertz

\author{mateuslumertz}
\definecolor{cor1}{RGB}{0,100,166}
\definecolor{cor2}{RGB}{100,195,213}
\definecolor{cor5}{RGB}{150,203,226}
\definecolor{cor3}{RGB}{30,130,186}
\definecolor{cor4}{RGB}{40,185,218}
\definecolor{preto}{RGB}{0,0,0}
\definecolor{branco}{RGB}{255,255,255}
% Configuração das Cores
\setbeamercolor{paleta1}{fg=cor1,bg=white}
\setbeamercolor{paleta2}{fg=cor1,bg=white}
\setbeamercolor{estrutura}{fg=cor1,bg=white}
\setbeamercolor{titulo_rodape}{fg=black,bg=white}
\setbeamercolor{data_rodape}{fg=gray,bg=white}
\setbeamercolor{frametitle}{fg=branco,bg=cor1}

% Modelo do rodapé
\defbeamertemplate*{footline}{mytheme}{%
  \leavevmode%
  \hbox{\begin{beamercolorbox}[wd=.5\paperwidth,ht=2.5ex,dp=1.125ex,leftskip=.3cm,rightskip=.3cm]{titulo_rodape}%
    \makebox[2em][l]{{\usebeamerfont{titulo_rodape}\textcolor{cor1}{\insertframenumber}}}%
    {\usebeamercolor{titulo_rodape}\insertshorttitle}
  \end{beamercolorbox}%
  \begin{beamercolorbox}[wd=.2\paperwidth,ht=2.5ex,dp=1.125ex,leftskip=.3cm,rightskip=.3cm]{data_rodape}%
    \usebeamerfont{data_rodape}\insertshortdate%
  \end{beamercolorbox}%
  \begin{beamercolorbox}[wd=.3\paperwidth,ht=2.5ex,dp=1.125ex,leftskip=.3cm,rightskip=.3cm,right]{titulo_rodape}%
    \includegraphics[width=.2\paperwidth,height=7ex,keepaspectratio]{imagens/logo.png}\hspace*{2em}%
  \end{beamercolorbox}}%
  \vskip0pt%
}

% Slide de Título
\defbeamertemplate*{title page}{mytheme}[1][]
{%
  \begin{tikzpicture}[remember picture,overlay]
    \filldraw[cor1]
    (current page.north west) --
    ([yshift=-12cm]current page.north west) --
    ([xshift=-4cm,yshift=-12cm]current page.north east) {[rounded corners=15pt]--
    ([xshift=-4cm,yshift=3cm]current page.south east)} --
    ([yshift=3cm]current page.south west) --
    (current page.south west) --
    (current page.south east) --
    (current page.north east) -- cycle
    ;
  \filldraw[branco]
    (current page.north west) --
    ([yshift=-2.15cm]current page.north west) --
    ([xshift=-3cm,yshift=-2.15cm]current page.north east) {[rounded corners=15pt]--
    ([xshift=-3cm,yshift=3cm]current page.south east)} --
    ([yshift=3cm]current page.south west) --
    (current page.south west) --
    (current page.south east) --
    (current page.north east) -- cycle
    ;
    \filldraw[cor2]
    ([xshift=-0.25cm,yshift=3cm]current page.south east)--
    ([xshift=-2.75cm,yshift=3cm]current page.south east)--
    ([xshift=-2.75cm,yshift=3.85cm]current page.south east)--
    ([xshift=-0.25cm,yshift=3.85cm]current page.south east)-- cycle
    ;
    \filldraw[cor3]
    ([xshift=-0.25cm,yshift=4cm]current page.south east)--
    ([xshift=-2.75cm,yshift=4cm]current page.south east)--
    ([xshift=-2.75cm,yshift=4.85cm]current page.south east)--
    ([xshift=-0.25cm,yshift=4.85cm]current page.south east)-- cycle
    ;
    \filldraw[cor4]
    ([xshift=-0.25cm,yshift=5cm]current page.south east)--
    ([xshift=-2.75cm,yshift=5cm]current page.south east)--
    ([xshift=-2.75cm,yshift=5.85cm]current page.south east)--
    ([xshift=-0.25cm,yshift=5.85cm]current page.south east)-- cycle
    ;
    \filldraw[cor5]
    ([xshift=-0.25cm,yshift=6cm]current page.south east)--
    ([xshift=-2.75cm,yshift=6cm]current page.south east)--
    ([xshift=-2.75cm,yshift=6.85cm]current page.south east)--
    ([xshift=-0.25cm,yshift=6.85cm]current page.south east)-- cycle
    ;
  \node[text=branco,anchor=south west,font=\sffamily\LARGE,text width=.68\paperwidth] 
  at ([xshift=10pt,yshift=-0.5cm]current page.west)
  (title)
  {\raggedright\inserttitle};  
  
  \node[text=cor1,anchor=south west,font=\sffamily\small,text width=.75\paperwidth] 
  at ([xshift=10pt,yshift=3.6cm]current page.west)
  (title)
  {\raggedright Université Côte d'Azur};  
  
  \node[anchor=east]
  at ([xshift=-0.15cm,yshift=-1cm]current page.north east)
  {\includegraphics[width=2.5cm]{imagens/logo.png}};
  
  \node[text=preto,font=\large\sffamily,anchor=south west]
  at ([xshift=30pt,yshift=0.5cm]current page.south west)
  (date)
  {\insertdate};
  \node[text=preto,font=\large\sffamily,anchor=south west]
  at ([yshift=5pt]date.north west)
  (author)
  {\insertauthor};
  \end{tikzpicture}%
}

% remove navigation symbols
\setbeamertemplate{navigation symbols}{}

% definition of the itemize templates
\setbeamertemplate{itemize item}[circle]
\setbeamercolor{itemize item}{fg=cor3,bg=white}
\setbeamercolor{itemize subitem}{fg=cor4,bg=white}
\setbeamercolor{itemize subsubitem}{fg=cor2,bg=white}

%%%%%%%%%%%%%%%%%%%%%%%%%%%%%%%%%%%%%%%%%%%%%%%%%%%%%%%%%%%%%%%%%%%%
%%%%%%%%%%%%%%%%%%%%%%%%%%%%%%%%%%%%%%%%%%%%%%%%%%%%%%%%%%%%%%%%%%%%
%%%%%%%%%%%%%%%%%%%%%%%%%%%%%%%%%%%%%%%%%%%%%%%%%%%%%%%%%%%%%%%%%%%%
%%%%%%%%%%%%%%%%%%%%%%%%%%%%%%%%%%%%%%%%%%%%%%%%%%%%%%%%%%%%%%%%%%%%
%%%%%%%%%%%%%%%%%%%%%%%%%%%%%%%%%%%%%%%%%%%%%%%%%%%%%%%%%%%%%%%%%%%%

\title[R Applications]{Analysis of Diamonds, 2013 NYC flights, and French Population using R}
\author{Quentin Le Roux}
\date{\today}

\begin{document}

\begin{frame}[plain]
\maketitle
\end{frame}

%Slide 
\begin{frame}
    \frametitle{Diamonds}
    \framesubtitle{Why are low-quality diamonds more expensive?}

    \begin{itemize}
        \item 1\% of the dataset only is ‘fair’ or ‘good’ diamonds under \$600 (Q4)
        \item Low clarity diamonds yield a higher price on average within the same color group (Q8)
    \end{itemize}
    \medskip
    $\Rightarrow$ \textbf{Intuition:} low quality diamonds generally never hit the market
    \medskip
    \begin{itemize}
        \item Low-carat (weight) diamonds are generally rare in the dataset (Q10)
        \item Positive linear dependence between carat and price (Q11, 12, 13)
        \item Carat is negative-correlated with cut and clarity
    \end{itemize}
    \medskip
    \textbf{Conclusion:} Diamonds of a lower quality yield a higher price because they are generally of a larger carat, which is positively correlated with price and usually overtake the negative effect of low quality
\end{frame}

%Slide 
\begin{frame}
    \frametitle{Flights in and out of NYC in 2013}
    \framesubtitle{What affects the number of daily flights?}
    Focus on the flights table among all available tables in the dataset.
    \medskip
    \begin{itemize}
        \item Whether it is a business day or a weekend day -- there is an increase in flights during the week, likely due to business travels (Q10, 11)
        \item Whether it is a school break or semester period -- holidays tend to increase the average number of flights (Q12, 13, 14)
        \item \textit{Exception:} during Christmas, there is a spike then a drop in the number of flights, likely due to increased traffic before Dec. 24th, then because of people being among family
    \end{itemize}
    \medskip
    \medskipweekly and school-dependent seasonality
    \medskip
    \textbf{Expanding the question:} Studying the weather to find epiphenomena (storms, etc.) could also be done
\end{frame}

%Slide 
\begin{frame}
    \frametitle{City population in France}
    \framesubtitle{}
    \textbf{Observations}
    \begin{itemize}
        \item Holes in the available dates
        \item France is a very centralized country (around Paris), projecting Paris' population will squish the other available data
    \end{itemize}
    Demo!
\end{frame}

\end{document}